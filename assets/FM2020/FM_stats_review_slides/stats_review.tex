% !TEX TS-program = pdflatex
% !TEX encoding = UTF-8 Unicode

% This file is a template using the "beamer" package to create slides for a talk or presentation
% - Giving a talk on some subject.
\documentclass{beamer}


\mode<presentation>
{
  \usetheme{Frankfurt}
  \usecolortheme{whale}
  % or ...

  \setbeamercovered{transparent}
  % or whatever (possibly just delete it)
}
\setbeamertemplate{headline}{}

\usepackage[english]{babel}
% or whatever
\usepackage{amsmath,amsfonts,amsthm,amssymb,mathrsfs}
\usepackage[utf8]{inputenc}
\usepackage{tikz}
% or whatever
\usepackage{natbib}
\usepackage{times}
\usepackage[T1]{fontenc}
\usepackage{blkarray}
\usepackage{algorithm}
\usepackage{algorithmic}
\DeclareMathOperator*{\argmin}{arg\,min}
\DeclareMathOperator*{\argmax}{arg\,max}
\DeclareMathOperator*{\argsup}{arg\,sup}
\newcommand\eqd {\overset{\mathrm{d}}{=}}
\newcommand{\eps}{\ensuremath{\varepsilon}}
\newcommand{\N}{\mathbb{N}} \newcommand{\Z}{\mathbb{Z}} \newcommand{\Q}{\mathbb{Q}} \newcommand{\R}{\mathbb{R}}
\newcommand{\s}{^*}
\renewcommand{\t}{^{\top}}
\newcommand{\inv}{^{-1}}
\newcommand{\E}{\mathbb{E}}
\newcommand{\p}{\mathbb{P}}
\newcommand{\ipq}{\mathbf{I}_{p,q}}
% Or whatever. Note that the encoding and the font should match. If T1
% does not look nice, try deleting the line with the fontenc.
%\newtheorem{theorem}{Theorem}


\title[Statistics Review] % (optional, use only with long paper titles)
{Statistics Review}

%\subtitle
%{} % (optional)

\author[Joshua Agterberg] % (optional, use only with lots of authors)
{Joshua Agterberg}
% - Use the \inst{?} command only if the authors have different
%   affiliation.
\institute[My Inst.]{\small Johns Hopkins University}


\date[Short Occasion] % (optional)
{\small Zoom}

\subject{Talks}
% This is only inserted into the PDF information catalog. Can be left
% out. 



% If you have a file called "university-logo-filename.xxx", where xxx
% is a graphic format that can be processed by latex or pdflatex,
% resp., then you can add a logo as follows:

% \pgfdeclareimage[height=0.5cm]{university-logo}{university-logo-filename}
% \logo{\pgfuseimage{university-logo}}



% Delete this, if you do not want the table of contents to pop up at
% the beginning of each subsection:
\AtBeginSection[]
{
  \begin{frame}<beamer>{Outline}
    \tableofcontents[currentsection]
  \end{frame}
} 

\setcounter{tocdepth}{1}


% If you wish to uncover everything in a step-wise fashion, uncomment
% the following command: 

%\beamerdefaultoverlayspecification{<+->}


\begin{document}

\begin{frame}
  \titlepage
\end{frame}

\begin{frame}{Outline}
  \tableofcontents
  % You might wish to add the option [pausesections]
\end{frame}


% Since this a solution template for a generic talk, very little can
% be said about how it should be structured. However, the talk length
% of between 15min and 45min and the theme suggest that you stick to
% the following rules:  

% - Exactly two or three sections (other than the summary).
% - At *most* three subsections per section.
% - Talk about 30s to 2min per frame. So there should be between about
%   15 and 30 frames, all told.
\section{Preliminaries}

\begin{frame}{Samples and Population}
\begin{itemize}
\item We have a population distribution $\mathcal{F} = \{ f: f \in \mathcal{F}\}$
\pause
\begin{itemize}
\item

\end{itemize}
\end{itemize}

\end{frame}


\section{Exact Parametric Methods}

\subsection{Estimation}

\subsection{One-Sample Parametric Testing}

\subsection{Two-Sample Parametric Testing}
 
 \section{Large-Sample Parametric Methods}
 
 \subsection{Estimation}
 
 \subsection{Testing}


\section{Linear Regression}

\subsection{Simple Linear Regression}

\subsection{Variable Selection Methods}

\section{Nonparametric Techniques}


\section{Machine Learning}
\subsection{Supervised Learning}

\subsection{Unsupervised Learning}
\subsubsection{Dimensionality Reduction}
\subsubsection{Clustering}



\begin{frame}[allowframebreaks]{References}
\bibliographystyle{plainnat_JA}
\end{frame}




\end{document}


